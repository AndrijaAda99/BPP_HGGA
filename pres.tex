%-----------------------------------------------------------------------------
%	PACKAGES AND THEMES
%-----------------------------------------------------------------------------
\documentclass[aspectratio=169,xcolor=dvipsnames]{beamer}
\usetheme{SimplePlus}

\usepackage[utf8]{inputenc}
\usepackage{hyperref}
\usepackage{graphicx}
\usepackage{booktabs}
\usepackage[serbian]{babel}
\usepackage{pgfplots}


%-----------------------------------------------------------------------------
%	TITLE PAGE
%-----------------------------------------------------------------------------

\title[short title]{Genetski algoritam za minimalno pakovanje}
\subtitle{Hibridni genetski algoritam grupa za problem jednodimenzionog minimalno pokovanje}

\author[andrija] {Andrija Urošević}

\institute[matf]
{%
    Univerzitet u Beogradu\\
    Matematički fakultet
}
\date{Septembar, 2022.}


%-----------------------------------------------------------------------------
%	PRESENTATION SLIDES
%-----------------------------------------------------------------------------

\begin{document}

\begin{frame}
    \titlepage%
\end{frame}

\begin{frame}{Pregled}
    \tableofcontents
\end{frame}

%-----------------------------------------------------------------------------
\section{Uvod --- Problem minimalnog pakovanja}
%-----------------------------------------------------------------------------

\begin{frame}{Problem minimalnog pakovanja}
    \begin{block}{Definicija}
        Neka je dat konačan skup elemenata $I$, neka svaki element $i \in I$ 
        ima veličinu $s(i)$, i neka je dat kapacitet 
        kontejnera $B$. Treba naći particiju $\{I_1, I_2, \ldots, I_k\}$ skupa 
        elemenata $I$, tako da je $\sum_{i \in I_k} s(i) \leq B$ i broj 
        kontejnera $k$ najmanji moguć.
    \end{block}
    \begin{exampleblock}{NP-težak}
        Problem minimalnog pakovanja je NP-težak.
    \end{exampleblock}
    \begin{block}{Problem odlučivanja}
        Da li za broj kontejnera $k$ postoji particija 
        $\{I_1, I_2, \ldots, I_k\}$ skupa elemenata $I$, 
        tako da $\sum_{i \in I_k} s(i) \leq B$?
    \end{block}
\end{frame}

%-----------------------------------------------------------------------------

\begin{frame}[t]{Jednodimenzioni problem minimalnog pakovanja}
    \begin{block}{Ograničenja jednodimenzione varijante problema}
        Veličine elemenata $s(i) \in \mathbb{Z}^{+}$ i
        kapacitet kontejnera $B \in \mathbb{Z}^{+}$.
    \end{block}
    \begin{block}{$OPT(I)$}
        Optimalan vrednost broj kontejnera $k$ za dati skup elemenata $I$, 
        obeležavamo i sa $OPT(I)$.
    \end{block}
    \begin{exampleblock}{Najbolje polinomijalne aproksimacije}
        \begin{enumerate}
            \item [FFD:] $\frac{3}{2} OPT(I)$
            \item [MFFD:] $\frac{71}{60} OPT(I)$ (1995)
            \item [KK:] $OPT(I) + O(\log^2 (OPT(I)))$ (1982)
            \item [HB:] $OPT(I) + O(\log(OPT(I))\log\log(OPT(I)))$ (2013)
            \item [HB:] $OPT(I) + O(\log(OPT(I)))$ (2017)
        \end{enumerate}
    \end{exampleblock}
\end{frame}

%-----------------------------------------------------------------------------

\begin{frame}{Pregled}
    \tableofcontents
\end{frame}

%-----------------------------------------------------------------------------
\section{Genetski algoritam za minimalno pakovanje}
%-----------------------------------------------------------------------------

\begin{frame}{Naivni pristup}
    \begin{block}{Naivni hromozom}
        Hromozom kod naivnog pristupa podrazumeva listu dužine $\|I\|$, 
        gde je $i$-ti element u listi kontejner u koji pakujemo 
        element $i \in I$.
    \end{block}
    \begin{exampleblock}{Primer i dedudandnost}
        \[
            \begin{aligned}
                ABCCAC \\
                CBAACA
            \end{aligned}
        \]
    \end{exampleblock}
\end{frame}

%-----------------------------------------------------------------------------

\begin{frame}{Permutacije}
    \begin{block}{Hromozom kao permutacija}
        Hromozom predstavlja permutaciju elemenata tako da se pakovenje 
        dobija dekodiranjem te permutacije. Mehanizam dekodiranja predstavlja 
        dodavanje elemenata u kontejner sve dok je to moguće, ukoliko nije 
        otvara se novi kontejner.
    \end{block}
    \begin{exampleblock}{Primer i redudandnost}
        \[
            \begin{aligned}
                0123|45678|9 \\
                3210|45678|9
            \end{aligned}
        \]
    \end{exampleblock}
\end{frame}

%-----------------------------------------------------------------------------

\begin{frame}{Grupe}
    \begin{block}{Hromozom: lista grupa}
        Reprezentacija hromozoma predstavlja listu grupa elemenata, 
        tj.\ listu kontejnera nekog dopustivog pakovanja.
    \end{block}
    \begin{exampleblock}{Primer}
        \[
            \begin{aligned}
                A=\{0, 4\}\ B=\{1\}\ C=\{2, 3, 5\} \\
                \{0, 4\}\ \{1\}\ \{2, 3, 5\}
            \end{aligned}
        \]
    \end{exampleblock}
\end{frame}

%-----------------------------------------------------------------------------

\begin{frame}{Operator ukrštanja}
    \begin{exampleblock}{Primer roditelja}
        \[
            \begin{aligned}
                \{0, 4\}\ \{1\}\ \{2, 3, 5\} \\
                \{1\}\ \{3\}\ \{2\}\ \{0, 4, 5\}
            \end{aligned}
        \]
    \end{exampleblock}
\end{frame}

%-----------------------------------------------------------------------------

\begin{frame}{Operator ukrštanja}
    \begin{exampleblock}{Primer roditelja}
        \[
            \begin{aligned}
                \{0, 4\}\ \{1\}\ \{2, 3, 5\} \\
                \{1\}\ \{3\}\ \{2\}\ \{0, 4, 5\}
            \end{aligned}
        \]
    \end{exampleblock}
    \begin{exampleblock}{Faza I:\ Segmentacija}
        \[
            \begin{aligned}
                \{0, 4\}\ |\ \{1\}\ |\ \{2, 3, 5\} \\
                \{1\}\ |\ \{3\}\ \{2\}\ |\ \{0, 4, 5\}
            \end{aligned}
        \]
    \end{exampleblock}
\end{frame}

%-----------------------------------------------------------------------------

\begin{frame}{Operator ukrštanja}
    \begin{exampleblock}{Primer roditelja}
        \[
            \begin{aligned}
                \{0, 4\}\ \{1\}\ \{2, 3, 5\} \\
                \{1\}\ \{3\}\ \{2\}\ \{0, 4, 5\}
            \end{aligned}
        \]
    \end{exampleblock}
    \begin{exampleblock}{Faza I:\ Segmentacija}
        \[
            \begin{aligned}
                \{0, 4\}\ |\ \{1\}\ |\ \{2, 3, 5\} \\
                \{1\}\ |\ \{3\}\ \{2\}\ |\ \{0, 4, 5\}
            \end{aligned}
        \]
    \end{exampleblock}
    \begin{exampleblock}{Faza II:\ Umetanje}
        \[
            \begin{aligned}
                \{0, 4\}\ |\ \{1\}\ |\ \{3\}\ \{2\}\ |\ \{2, 3, 5\} \\
                \{1\}\ |\ \{1\}\ |\ \{3\}\ \{2\}\ |\ \{0, 4, 5\}
            \end{aligned}
        \]
    \end{exampleblock}
\end{frame}

%-----------------------------------------------------------------------------

\begin{frame}{Operator ukrštanja}
    \begin{exampleblock}{Faza I:\ Segmentacija}
        \[
            \begin{aligned}
                \{0, 4\}\ |\ \{1\}\ |\ \{2, 3, 5\} \\
                \{1\}\ |\ \{3\}\ \{2\}\ |\ \{0, 4, 5\}
            \end{aligned}
        \]
    \end{exampleblock}
    \begin{exampleblock}{Faza II:\ Umetanje}
        \[
            \begin{aligned}
                \{0, 4\}\ |\ \{1\}\ |\ \{3\}\ \{2\}\ |\ \{2, 3, 5\} \\
                \{1\}\ |\ \{1\}\ |\ \{3\}\ \{2\}\ |\ \{0, 4, 5\}
            \end{aligned}
        \]
    \end{exampleblock}
    \begin{exampleblock}{Faza III:\ Brisanje duplikata}
        \[
            \begin{aligned}
                \text{izbačeni}= \{ 5\};\ \{0, 4\}\ |\ \{1\}\ |\ \{3\}\ \{2\}\ |\  \\
                \text{izbačeni}= \{\};\ |\ \{1\}\ |\ \{3\}\ \{2\}\ |\ \{0, 4, 5\}  
            \end{aligned}
        \]
    \end{exampleblock}
\end{frame}

%-----------------------------------------------------------------------------

\begin{frame}{Operator ukrštanja}
    \begin{exampleblock}{Faza II:\ Umetanje}
        \[
            \begin{aligned}
                \{0, 4\}\ |\ \{1\}\ |\ \{3\}\ \{2\}\ |\ \{2, 3, 5\} \\
                \{1\}\ |\ \{1\}\ |\ \{3\}\ \{2\}\ |\ \{0, 4, 5\}
            \end{aligned}
        \]
    \end{exampleblock}
    \begin{exampleblock}{Faza III:\ Brisanje duplikata}
        \[
            \begin{aligned}
                \text{izbačeni}= \{ 5\};\ \{0, 4\}\ |\ \{1\}\ |\ \{3\}\ \{2\}\ |\  \\
                \text{izbačeni}= \{\};\ |\ \{1\}\ |\ \{3\}\ \{2\}\ |\ \{0, 4, 5\}  
            \end{aligned}
        \]
    \end{exampleblock}
    \begin{exampleblock}{Faza IV:\ Heuristika ubacivanja}
        \[
            \begin{aligned}
                \{0, 4, 5\}\ |\ \{1\}\ |\ \{3\}\ \{2\}\ |\  \\
                |\ \{1\}\ |\ \{3\}\ \{2\}\ |\ \{0, 4, 5\}  
            \end{aligned}
        \]
    \end{exampleblock}
\end{frame}

%-----------------------------------------------------------------------------

\begin{frame}{Operator ukrštanja}
    \begin{exampleblock}{Faza III:\ Brisanje duplikata}
        \[
            \begin{aligned}
                \text{izbačeni}= \{ 5\};\ \{0, 4\}\ |\ \{1\}\ |\ \{3\}\ \{2\}\ |\  \\
                \text{izbačeni}= \{\};\ |\ \{1\}\ |\ \{3\}\ \{2\}\ |\ \{0, 4, 5\}  
            \end{aligned}
        \]
    \end{exampleblock}
    \begin{exampleblock}{Faza IV:\ Heuristika ubacivanja}
        \[
            \begin{aligned}
                \{0, 4, 5\}\ |\ \{1\}\ |\ \{3\}\ \{2\}\ |\  \\
                |\ \{1\}\ |\ \{3\}\ \{2\}\ |\ \{0, 4, 5\}  
            \end{aligned}
        \]
    \end{exampleblock}
    \begin{exampleblock}{Faza V:\ Naslednici}
        \[
            \begin{aligned}
                \{0, 4, 5\}\ \{1\}\ \{3\}\ \{2\}  \\
                \{1\}\ \{3\}\ \{2\}\ \{0, 4, 5\}  
            \end{aligned}
        \]
    \end{exampleblock}
\end{frame}

%-----------------------------------------------------------------------------

\begin{frame}{Operator mutacije}
    \begin{block}{Definicija}
        \begin{enumerate}
            \item Uništi neke od grupa.
            \item Primeni fazu IV:\ Heuristika ubacivanja.
        \end{enumerate}
    \end{block}
\end{frame}

%-----------------------------------------------------------------------------

\begin{frame}{Funkcija prilagođenosti}
    \begin{block}{Funkcija prilagođenosti}
        \begin{equation}
            \text{fitness} = \frac{\sum_{j = 1}^N {(F_j/B)}^k}{N},
        \end{equation}
        gde je $N$ broj grupa u hromozomu, $B$ je kapacitet kontejnera, $F_j$
        je napunjenost kontejnera $j$, i $k > 1$ je predefinisani koeficijent 
        koji pojačava odnose dobre upakovanosti
    \end{block}
    \begin{block}{Napunjenost kontejnera $j$}
        \begin{equation}
            F_j = \sum_{i \in I_j} s(i),
        \end{equation}
        gde je je $I_j$ skup elemenata iz grupe $j$, a $s(i)$ veličina 
        elementa $i$.
    \end{block}
\end{frame}

%-----------------------------------------------------------------------------

\begin{frame}{Ostali parametri genetskog algoritma}
    \begin{itemize}
        \item verovatnoća mutacije: 0.66
        \item verovatnoča ukrštanja: 1.00
        \item broj jedinki ukrštanja: 2
        \item broj naslednika: 2
        \item tip zamene: roditelje menjaju naslednici
        \item veličina populacije: 100
        \item sortirana populacija: da
        \item kriterijum sortiranja: maksimizacija funkcije prilagođenosti
        \item broj selekcija: $2 \times 50$
        \item tip selekcije: turnirksa
        \item broj turnira: 5
        \item broj iteracija: 50
        \item kriterijum zaustavljanja: broj iteracija
    \end{itemize}
\end{frame}

%-----------------------------------------------------------------------------

\begin{frame}{Pregled}
    \tableofcontents
\end{frame}

%-----------------------------------------------------------------------------
\section{Rezultati i Zaključajk}
%-----------------------------------------------------------------------------

\begin{frame}
    \Huge{\centerline{\textbf{Rezultati}}}
\end{frame}

%-----------------------------------------------------------------------------

\begin{frame}{Zaključak i dalji rad}
    \begin{block}{Zaključak}
        \begin{itemize}
            \item Vremenski efikasno rešava čak i teške instance.
            \item Na nekim instancama rešenje je daleko od optimuma.
        \end{itemize}
    \end{block}
    \begin{block}{Dalji rad}
        \begin{itemize}
            \item Probati razne heuristike u fazi IV.\
            \item Probati razne parametre genetskog algoritma.
        \end{itemize}
        
    \end{block}
\end{frame}

%-----------------------------------------------------------------------------

\begin{frame}
    \Huge{\centerline{\textbf{Pitanja?}}}
\end{frame}

%-----------------------------------------------------------------------------

\end{document}
