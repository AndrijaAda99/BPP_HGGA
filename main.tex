\documentclass[a4paper,12pt,twocolumn]{article}
\usepackage[serbian]{babel}
\usepackage[T2A]{fontenc} 
\usepackage[utf8]{inputenc} 
\usepackage[margin=1in]{geometry}
\usepackage{multicol} 
\usepackage{csquotes}
\usepackage[nottoc]{tocbibind}
\usepackage[backend=biber]{biblatex}
\usepackage{authblk}
\usepackage{amsfonts}

\addbibresource{main.bib}

\title{Minimal Bin Packing}
\author{Andrija Urošević\\\textit{Univerzitet u Beogradu, Matematicki fakultet}\\\texttt{andrija.urosevic@protonmail.com}}
\date{Decembar 2021.}

\begin{document}

\maketitle

\begin{abstract}
    Ovde ide Sažetak
\end{abstract}

\section{Uvod}

\emph{Problem minimalnog pakovanja u prostoru} 
(engl.\ Minimal Bin Packing Problem) podrazumeva pokovanje objekata 
različitih zapremina u konačno mnogo kontejnera sa ciljem da se minimizuje
broj potrebnih kontejnera za skladištenje datih objekata. Formalno, neka je
dat konačan skup elemenata $I$, neka svaki element $i \in I$ ima veličinu 
$s(i) \in \mathbb{Z}^{+}$, i neka je dat kapacitet kontejnera $B$. Treba naći
particiju $\{I_1, I_2, \ldots, I_k\}$ skupa elemenata $I$, tako da je 
$\sum_{i \in I_k} s(i) \leq B$ i broj kontejnera $k$ najmanji moguć. 
Ovaj problem je $NP$-težak\cite{gj80}. Njegova varijanta problema odlučivanja:
Da li za broj kontejnera $k$ postoji particija $\{I_1, I_2, \ldots, I_k\}$
skupa elemenata $I$, tako da $\sum_{i \in I_k} s(i) \leq B$? 

\emph{Jednodimenzioni problem minimalnog pakovanje} je problem minimalnog 
pakovanja u prostoru gde su veličine elemenata $s(i) \in \mathbb{Z}$ i 
gde je kapacitet kontejnera $B \in \mathbb{Z}$. Optimalan vrednost broj 
kontejnera $k$ za dati skup elemenata $I$, obeležavamo i sa $OPT(I)$.
U literaturi se koristi i normalizovani oblik problema. Naime, $B = 1$ i
$s(i) \in [0, 1]$ za svaki element $i \in I$ (normalizujemo instance tako što
podelimo veličinu $s(i)$ sa $B$, a $B$ postavimo na $1$). 

Problem jednodimenzionog minimalnog pakovanja napadamo
koristeći genetski algoritam sa specijalizovanom reprezentacijom i 
specijalizovanim operatorima genetskog algoritma. Trenutno najbolja 
polinomijalna aproksimativna rešenja upadaju u $\frac{3}{2} OPT(I)$\cite{sl94} 
i $\frac{71}{60} OPT(I)$\cite{jg85, yz95}. Ovaj rad, eksperimentalno, pokušava 
da upadne u ove granice.

\section{Genetski algoritam za minimalno pokovanje}

\subsection{Reprezentacija jedinke}

\subsection{Operator ukrštanja}

\subsection{Operator mutacija}

\section{Rezultati}

Algoritam je pokrenut na računaru sa sledećim specifikacijama: 
CPU --- Intel Core i5--2520M @ $4 \times 3.2GHz$,
RAM --- 7836MiB,
Kernel --- x86\_64 Linux 5.16.15-arch1--1,
g++ --- (GCC) 11.2.0.



\section{Zaključak}

Ovo je zakljucak.

\nocite{*}

\printbibliography[title={Literatura}]

\end{document}
