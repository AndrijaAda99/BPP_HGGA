\documentclass[a4paper,12pt,twocolumn]{article}
\usepackage[serbian]{babel}
\usepackage[T2A]{fontenc} 
\usepackage[utf8]{inputenc} 
\usepackage[margin=1in]{geometry}
\usepackage{multicol} 
\usepackage{csquotes}
\usepackage{float}
\usepackage{hyperref}
\usepackage[nottoc]{tocbibind}
\usepackage[backend=biber]{biblatex}
\usepackage{authblk}
\usepackage{amsmath}
\usepackage{amsfonts}
\usepackage{algorithm}
\usepackage{algpseudocode}
\usepackage{ifthen,array,booktabs}
\usepackage{csvsimple-l3}



\makeatletter
\renewcommand*{\ALG@name}{Algoritam}
\makeatother

\addbibresource{main.bib}

\title{Hibridni genetskom algoritam grupa u rešavanju jednodimenzionog problema minimalnog pakovanja}
\author{Andrija Urošević\\\textit{Univerzitet u Beogradu, Matematicki fakultet}\\\texttt{andrija.urosevic@protonmail.com}}
\date{Septembar 2022.}

\begin{document}

\maketitle

\begin{abstract}
    Genetski algoritmi imaju široku primenu u oblasti optimizacije. Specijalno
    koriste se za probleme rapsoređivanja i planiranja. Problem minimalnog 
    pakovanje je jedan takav problem koji je NP-težak. Ovaj rad istražuje 
    moguće implementacije genetskog algoritma, i potencijalne probleme koji 
    nastaju pri implementaciji. Tačnije, kako najbolje predstaviti jedinke
    i kako definisati operatore ukrštanja i mutacije, zajedno sa odgovarajućom
    funkcijom prilagođenosti. Operatori ukrštanja i mutacije sadrže dodatne
    heuristike pa se zbog toga genetski algoritam nazivamo hibridnim. Takođe,
    postavlja se pitanje kako definisati jedinke, te se uvodi koncept grupa.
    Zbog toga ovaj algoritam ima naziv hibridni genetski algoritam grupa.
\end{abstract}

\section{Uvod}

\emph{Problem minimalnog pakovanja u prostoru} 
(engl.\ Minimal Bin Packing Problem) podrazumeva pokovanje objekata 
različitih zapremina u konačno mnogo kontejnera sa ciljem da se minimizuje
broj potrebnih kontejnera za skladištenje datih objekata. Formalno, neka je
dat konačan skup elemenata $I$, neka svaki element $i \in I$ ima veličinu 
$s(i) \in \mathbb{Z}^{+^n}$, i neka je dat kapacitet kontejnera $B$. Treba naći
particiju $\{I_1, I_2, \ldots, I_k\}$ skupa elemenata $I$, tako da je 
$\sum_{i \in I_k} s(i) \leq B$ i broj kontejnera $k$ najmanji moguć. 
Ovaj problem je $NP$-težak\cite{gj80}. Njegova varijanta problema odlučivanja:
Da li za broj kontejnera $k$ postoji particija $\{I_1, I_2, \ldots, I_k\}$
skupa elemenata $I$, tako da $\sum_{i \in I_k} s(i) \leq B$? 

\emph{Jednodimenzioni problem minimalnog pakovanje} je problem minimalnog 
pakovanja u prostoru gde su veličine elemenata $s(i) \in \mathbb{Z}^{+}$ i 
gde je kapacitet kontejnera $B \in \mathbb{Z}^{+}$. Optimalan vrednost broj 
kontejnera $k$ za dati skup elemenata $I$, obeležavamo i sa $OPT(I)$.
U literaturi se koristi i normalizovani oblik problema. Naime, $B = 1$ i
$s(i) \in [0, 1]$ za svaki element $i \in I$ (normalizujemo instance tako što
podelimo veličinu $s(i)$ sa $B$, a $B$ postavimo na $1$). 

Problem jednodimenzionog minimalnog pakovanja napadamo
koristeći genetski algoritam sa specijalizovanom reprezentacijom i 
specijalizovanim operatorima genetskog algoritma. Trenutno najbolja 
polinomijalna aproksimativna rešenja upadaju u $\frac{3}{2} OPT(I)$\cite{sl94} 
i $\frac{71}{60} OPT(I)$\cite{jg85, yz95}. Ovaj rad, eksperimentalno, pokušava 
da upadne u ove granice. Postoje i bolje, kompleksnije polinomijalne 
aproksimacije čije rešenje nije veće od 
$OPT(I) + O(\log^2 (OPT(I)))$\cite{kk82}. Dalje imamo poboljšanja ove ideje
gde rešenje nije veće od
$OPT(I) + O(\log(OPT(I))\log\log(OPT(I)))$\cite{r13} i na kraju
$OPT(I) + O(\log(OPT(I)))$\cite{hr17}.


\section{Genetski algoritam za minimalno pokovanje}

Problem jednodimenzionog minimalnog pakovanja pokušavamo da rešimo genetskim
algoritmom. Postoje mnoga pitanja i problemi koji nastaju pri implementiranju 
genetskog algoritma za ovaj problem. Glavni problemi koji nastaju je genetski
algoritam koji se ponaša kao nasumična pretraga\cite{f96}. Naime, izbor 
reprezentacija jedinki, a kasnije i odgovarajućih specijalizovanih operatora
ukrštanja i mutacije je od ključnog značaja za rešavanje ovog problema. 
U daljem tekstu su opisani načini na koje treba implementirati genetski 
algoritam za efikasno rešavanja problema jednodimenzionog pakovanja.

\subsection{Reprezentacija jedinki}

Razmatramo tri načina za predstavljanje hromozoma: Naivni pristup, 
permutacije i grupe. Svaki sledeći, respektivno, smanjuje nivo redudandnosti
koja se javljaju u reprezentaciji hromozoma.

\subsubsection{Naivni pristup}

Hromozom kod naivnog pristupa podrazumeva listu dužine $\|I\|$, gde je $i$-ti
element u listi kontejner u koji pakujemo element $i \in I$. Na primer, 
hromozom
\[ABCCAC\]
predstavlja pakovanje gde se prvi element pakuje u kontejner $A$, drugi 
element se pakuje u kontejner $B$, dok se treći i četvrti element pakuju u 
kontejner $C$.

Ovaj pristup sa sobom nosi puno redudandnosti. Naime, sledeća dva hromozoma
enkodiraju istu informaciju:
\[
    \begin{aligned}
        ABCCAC; \\
        CBAACA.
    \end{aligned}
\]

\subsubsection{Permutacije}

Radi smanjenja redudandnosti veoma često se koriste permutacije. Neka
su elementi $I$ obeleženi rednim brojevima. Hromozom predstavlja permutaciju
elemenata tako da se pakovenje dobija dekodiranjem te permutacije. 
Mehanizam dekodiranja za problem minimalnog pakovanje predstavlja dodavanje
elemenata u kontejner sve dok je to moguće, ukoliko nije otvara se novi 
kontejner.

Na primer, neka je data permutacije $0123456789$. Neka je tada jedno 
validno pakovanje 
\[0123|45678|9.\]

Jasno je da je i permutacija $3210456789$ enkodira isto pakovanje, tj.\ 
odgovarajuće pakovanje je oblika
\[3210|45678|9.\]

\subsubsection{Grupe}

Reprezentacija hromozoma predstavlja listu grupa elemenata, tj.\ listu
kontejnera nekog dopustivog pakovanja. Na primer, razmotrimo primer pakovanja
kod naivnog pristupa $ABCCAC$. Njegovo ekvivalentno pakovanje predstavljeno
hromozomom grupa se enkodira kao:
\[
    A=\{0, 4\},\ B=\{1\},\ C=\{2, 3, 5\}.
\]
Ovo možemo zapisati i kao:
\[
    \{0, 4\},\ \{1\},\ \{2, 3, 5\}.
\]

Ovim pristupom smanjujemo stepen redudandnosti koji nastaje i kod naivnog 
pristupa i kod permutacija. Te će genetski algoritam implementirati grupe
kao jedinke.

\subsection{Operator ukrštanja}

Operator ukrštanja PMX za hromozome predstavljenje pomoću permutacije ne
daje smislene naslednike\cite{f96}. Zbog toga je najbolje predstaviti 
hrozomozome preko grupa. Uobičajeni operatori ukrštanja nije moguće primeniti
na grupe, kako pri razmenjivanju grupa dobijamo nedopustivo rešenje. 
Zbog toga treba osmisliti specijalizovani operator ukrštanja.

\begin{algorithm}
    \caption{Operator ukrštanja}\label{alg:cross}
    \begin{algorithmic}[1]
        \Require\ $X, Y$ \Comment{Hromozomi roditelja}
        \State\ $(Start_X, End_X) \gets \mathcal{U}^2(0, \|X\|)$
        \State\ $(Start_Y, End_Y) \gets \mathcal{U}^2(0, \|Y\|)$
        \State\ $X \gets \Call{insert}{X, Y[Start_Y, End_Y]}$
        \State\ $Y \gets \Call{insert}{Y, X[Start_X, End_X]}$
        \State\ $X, B_X \gets \Call{remove duplcates}{X}$
        \State\ $Y, B_Y \gets \Call{remove duplcates}{Y}$
        \State\ $X \gets \Call{pack}{X, B_X}$
        \State\ $Y \gets \Call{pack}{Y, B_Y}$
    \end{algorithmic}
\end{algorithm}

Operaciju ukrštanja izvodimo po algoritmu~\ref{alg:cross}. Algoritam 
započinje tako što bira granice segmenata koji će se ukrštati 
$(Start_X, End_X)$ i $(Start_Y, End_Y)$ za hromozom $X$ i $Y$, respektivno. 
Nakon toga, vrši se umetanje segmetna $Y[Start_Y, End_Y]$ u $X$, nakon 
pozicije $End_X$, dok segment $X[Start_X, End_X]$ umetamo u $Y$ ispred 
poziciji $Start_Y$. Sada hromozomi u sebi sadrže duplikate, tj. elemente
koji se nalaze u umetnutim segmentima zajedno sa elementima drugih kontejnera.
Sledeći korak podrazumeva uklanjanje svih kontejnera koji u sebi sadrže
elemente iz umetnutog segmenta. Pri uklanjanju ovih kontejnera potencijalno
smo izbacili i one koji nisu bili duplikati. Njih smeštamo u skup $B_X$ i 
$B_Y$, respektivno u odnosu na hromozome $X$ i $Y$. Sada je potrebno odrediti 
novo pakovanje dodavanjem izbačenih elemenata u grupe iz hromozoma. Ako
dodavanje nije moguće ni za jednu grupu otvaramo novu grupu i dodajemo je u
hromozom. 

Ovim postupkom smo opisali opšti oblik specijalizovanog operatora ukrštanja
za hromozome koji su predstavljeni listom grupa. Treba još definisati na koji
način vršimo ubacivanje izbačenih elemenata i tako dobijamo novo pakovanje,
tj.\ dopustivo rešenje. Postoje mnoge efikasne heuristike koje to rade
u vremenskom ograničenju $O(n \log n)$. Na primer, FFD (engl.\ First-Fit 
Descending), NFD (engl.\ Next-Fit Descending), MFFD (engl.\ Modified First-Fit
Descending)\cite{cgj96}. Ovaj radi koristi FFD, koji sortira izbačene elemente po
opadajućoj veličini i tako sortirane ubacije jedan po jedan u prvu 
odgovarajuću grupu u koju može da se upakuje. Ako ne može da se upakuje ni u
jednu grupu, otvara se nova grupa u koju se ubacuje.

\subsection{Operator mutacija}

Slično, kao i kod operatora ukrštanja, treba odrediti specijalizovani operator
mutacije. Jedan način definisanja operatora mutacije se sastoji u tome da
iz hromozoma sa nekom verovatnoćom izbacujemo grupe. Izbačene elemente onda 
pakujemo nazad nekom heuristikom isto kao i kod operatora ukrštanja.

\subsection{Računanje prilagodjenosti}

Kako problem predstavlja minimizaciju broja potrebnih kontejnera za pakovanje,
prirodno je za ocenu prilagodjenosti izabrati broj grupa u hromozomu.
Problem nastaje kada dve jedinke imaju istu ocenu prilagodjenosti, tj.\ 
broj kontejnera je isti za oba pakovanja. Kako odrediti koje pakovanje je
bolje od ta dva? Ovo prevazilazimo tako što merimo koliko je svaki od 
kontejnera dobro upakovan. Naime, dobro upakovan kontejner je onaj koji
ima malo neiskorišćenog kapaciteta.

Formalno, prilagodjenost računamo kao
\begin{equation}
    \text{fitness} = \frac{\sum_{j = 1}^N {(F_j/B)}^k}{N},
\end{equation}
gde je $N$ broj grupa u hromozomu, $B$ je kapacitet kontejnera, $F_j$
je napunjenost kontejnera $j$, i $k > 1$ je predefinisani koeficijent koji
pojačava odnose dobre upakovanosti. Eksperimentalo $k=2$ daje dobre 
rezultate\cite{f96}. 

Napunjenost kontejnera $j$ računamo kao 
\begin{equation}
    F_j = \sum_{i \in I_j} s(i),
\end{equation}
gde je je $I_j$ skup elemenata iz grupe $j$, a $s(i)$ veličina elementa $i$.

Kako ovako definisana funkcija prilagođenosti predstavlja meru dobre
upakovanosti, i kako se problem svodi na određivanje minimalnog broja potrebnih 
kutija, onda će genetski algoritam vršiti maksimizaciju funkcije 
prilagođenosti.

\subsection{Ostali parametri genetskog algoritma}

\begin{itemize}
    \item verovatnoća mutacije: 0.66
    \item verovatnoča ukrštanja: 1.00
    \item broj jedinki ukrštanja: 2
    \item broj naslednika: 2
    \item tip zamene: roditelje menjaju naslednici
    \item veličina populacije: 100
    \item sortirana populacija: da
    \item kriterijum sortiranja: maksimizacija funkcije prilagođenosti
    \item broj selekcija: $2 \times 50$
    \item tip selekcije: turnirksa
    \item broj turnira: 5
    \item broj iteracija: 50
    \item kriterijum zaustavljanja: broj iteracija
\end{itemize}

\section{Rezultati}

Algoritam je pokrenut na računaru sa sledećim specifikacijama: 
CPU --- Intel Core i5--2520M @ $4 \times 3.2GHz$,
RAM --- 7836MiB,
Kernel --- x86\_64 Linux 5.16.15-arch1--1,
g++ --- (GCC) 11.2.0.

Za ekspreimentalno testiranje koristimo ručno napravljene lake
instance iz grupe EZ.\ Za dalje testiranje koristmo N1, N2 i 
H instance i njihove optimalne vrednosti koje su dobijene pomoću 
tačnog algoritma BISON\cite{skj97}.

\subsection{$EZ$ instance}

\csvnames{ez instances}{name=\instance,alg=\alg,sol=\sol,time=\time}

\csvstyle{ez table}{ez instances,
  centered tabular=crr,
  table head=\multicolumn{3}{c}{\bfseries #1}\\\toprule
    Instanca & Rezultat & Vreme($ms$) \\\midrule,
  late after line=\\, late after last line=\\\bottomrule}

\csvreader[ez table=Hibridni genetski algoritam grupa]{res/resEZ_GA.csv}{}{%
    \instance & \sol & \time 
}

\csvreader[ez table=Algoritam grube sile]{res/resEZ_BF.csv}{}{%
    \instance & \sol & \time 
}

Primećujemo da hibridni genetski algoritam grupa daje ista rešenja
kao i algoritam grube sile. Takođe vidimo da sa malim povećanjem
broja elemenata u instanci EZ9 vreme izvršavanja algoritma grube 
sile drastično povećava u odnosu na hibridni genetski algoritma 
grupa koji ima slično vreme izvršavanja. U ostalim instancama, 
zbog toga, ne pokrećemo algoritam grube sile, kako se neće 
završiti u realnom vremenu.

\subsection{N1 instance}

\csvnames{my names}{name=\instance,sol=\sol,opt=\opt,time=\time,diff=\diff,oopt=\oopt,odiff=\odiff}

\csvstyle{my table}{my names,
  centered tabular=crrr,
  table head=\multicolumn{4}{c}{\bfseries #1}\\\toprule
    Instanca & Rez. & Opt. & $t(ms)$ \\\midrule,
  late after line=\\, late after last line=\\\bottomrule}


\csvreader[my table=Neke N1 instanci]{res/resN1_part.csv}{}{%
    \instance & \sol & \opt & \time
}

\subsection{N2 instance}

\csvreader[my table=Neke N2 instanci]{res/resN2_part.csv}{}{%
    \instance & \sol & \opt & \time
}

\subsection{H instance}

\csvreader[my table=Teške instance]{res/resH_analysis.csv}{}{%
    \instance & \sol & \opt & \time
}


\section{Zaključak}

Hibridni genetski algoritam grupa primenjen na jednodimenzioni problem
minimalnog pakovanja daje zadovoljavajuće dobra rešenja u kratkom vremenskom
periodu, gde broj elemenata nije od ključanog faktor za vremensku efikasnost.
Zbog toga se može koristiti kao gornja granica za heuristike koje rešavaju
problem minimalnog pakovanja ili neki njegov dualni problem.

Dalji rad podrazumeva eksperimentalno testiranje drugih heuristika pri
ubacivanju izbačenih elemenata u grupe kod operatora ukrštanja i operatora
mutacije. Takođe, dalji rad može da uključi i podešavanje parametra genetskog 
algoritma.

\nocite{*}

\printbibliography[title={Literatura}]

\end{document}
