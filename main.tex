\documentclass[a4paper,12pt,twocolumn]{article}
\usepackage[serbian]{babel}
\usepackage[T2A]{fontenc} 
\usepackage[utf8]{inputenc} 
\usepackage[margin=1in]{geometry}
\usepackage{multicol} 
\usepackage{csquotes}
\usepackage[nottoc]{tocbibind}
\usepackage[backend=biber]{biblatex}
\usepackage{authblk}
\usepackage{amsmath}
\usepackage{amsfonts}

\addbibresource{main.bib}

\title{Genetskom algoritam hibridnih grupa u rešavanju jednodimenzionog problema minimalnog pakovanja}
\author{Andrija Urošević\\\textit{Univerzitet u Beogradu, Matematicki fakultet}\\\texttt{andrija.urosevic@protonmail.com}}
\date{Septembar 2022.}

\begin{document}

\maketitle

\begin{abstract}
    Ovde ide Sažetak
\end{abstract}

\section{Uvod}

\emph{Problem minimalnog pakovanja u prostoru} 
(engl.\ Minimal Bin Packing Problem) podrazumeva pokovanje objekata 
različitih zapremina u konačno mnogo kontejnera sa ciljem da se minimizuje
broj potrebnih kontejnera za skladištenje datih objekata. Formalno, neka je
dat konačan skup elemenata $I$, neka svaki element $i \in I$ ima veličinu 
$s(i) \in \mathbb{Z}^{+}$, i neka je dat kapacitet kontejnera $B$. Treba naći
particiju $\{I_1, I_2, \ldots, I_k\}$ skupa elemenata $I$, tako da je 
$\sum_{i \in I_k} s(i) \leq B$ i broj kontejnera $k$ najmanji moguć. 
Ovaj problem je $NP$-težak\cite{gj80}. Njegova varijanta problema odlučivanja:
Da li za broj kontejnera $k$ postoji particija $\{I_1, I_2, \ldots, I_k\}$
skupa elemenata $I$, tako da $\sum_{i \in I_k} s(i) \leq B$? 

\emph{Jednodimenzioni problem minimalnog pakovanje} je problem minimalnog 
pakovanja u prostoru gde su veličine elemenata $s(i) \in \mathbb{Z}$ i 
gde je kapacitet kontejnera $B \in \mathbb{Z}$. Optimalan vrednost broj 
kontejnera $k$ za dati skup elemenata $I$, obeležavamo i sa $OPT(I)$.
U literaturi se koristi i normalizovani oblik problema. Naime, $B = 1$ i
$s(i) \in [0, 1]$ za svaki element $i \in I$ (normalizujemo instance tako što
podelimo veličinu $s(i)$ sa $B$, a $B$ postavimo na $1$). 

Problem jednodimenzionog minimalnog pakovanja napadamo
koristeći genetski algoritam sa specijalizovanom reprezentacijom i 
specijalizovanim operatorima genetskog algoritma. Trenutno najbolja 
polinomijalna aproksimativna rešenja upadaju u $\frac{3}{2} OPT(I)$\cite{sl94} 
i $\frac{71}{60} OPT(I)$\cite{jg85, yz95}. Ovaj rad, eksperimentalno, pokušava 
da upadne u ove granice. Postoje i bolje, kompleksnije polinomijalne 
aproksimacije čije rešenje nije veće od 
$OPT(I) + O(\log^2 (OPT(I))$\cite{kk82}. Dalje imamo poboljšanja ove ideje
gde rešenje nije veće od
$OPT(I) + O(\log(OPT(I))\log\log(OPT(I)))$\cite{r13} i na kraju
$OPT(I) + O(\log(OPT(I)))$\cite{hr17}.


\section{Genetski algoritam za minimalno pokovanje}

Problem jednodimenzionog minimalnog pakovanja pokušavamo da rešimo genetskim
algoritmom. Postoje mnoga pitanja i problemi koji nastaju pri implementiranju 
genetskog algoritma za ovaj problem. Glavni problemi koji nastaju je genetski
algoritam koji se ponaša kao nasumična pretraga\cite{f96}. Naime, izbor 
reprezentacija jedinki, a kasnije i odgovarajućih specijalizovanih operatora
ukrštanja i mutacije je od ključnog značaja za rešavanje ovog problema. 
U daljem tekstu su opisani načini na koje treba implementirati genetski 
algoritam za efikasno rešavanja problema jednodimenzionog pakovanja.

\subsection{Reprezentacija jedinki}

Razmatramo tri načina za predstavljanje hromozoma: Naivni pristup, 
permutacije i grupe. Svaki sledeći, respektivno, smanjuje nivo redudandnosti
koja se javljaju u reprezentaciji hromozoma.

\subsubsection{Naivni pristup}

Hromozom kod naivnog pristupa podrazumeva listu dužine $\|I\|$, gde je $i$-ti
element u listi kontejner u koji pakujemo element $i \in I$. Na primer, 
hromozom
\[ABCCAC\]
predstavlja pakovanje gde se prvi element pakuje u kontejner $A$, drugi 
element se pakuje u kontejner $B$, dok se treći i četvrti element pakuju u 
kontejner $C$.

Ovaj pristup sa sobom nosi puno redudandnosti. Naime, sledeća dva hromozoma
enkodiraju istu informaciju:
\[
    \begin{aligned}
        ABCCAC; \\
        CBAACA.
    \end{aligned}
\]

\subsubsection{Permutacije}

Radi smanjenja redudandnosti veoma često se koriste permutacije. Neka
su elementi $I$ obeleženi rednim brojevima. Hromozom predstavlja permutaciju
elemenata tako da se pakovenje dobija dekodiranjem te permutacije. 
Mehanizam dekodiranja za problem minimalnog pakovanje predstavlja dodavanje
elemenata u kontejner sve dok je to moguće, ukoliko nije otvara se novi 
kontejner.

Na primer, neka je data permutacije $0123456789$. Neka je tada jedno 
validno pakovanje 
\[0123|45678|9.\]

Jasno je da je i permutacija $3210456789$ enkodira isto pakovanje, tj.\ 
odgovarajuće pakovanje je oblika
\[3210|45678|9.\]

\subsubsection{Grupe}

Reprezentacija hromozoma predstavlja listu grupa elemenata, tj.\ listu
kontejnera nekog dopustivog pakovanja. Na primer, razmotrimo primer pakovanja
kod naivnog pristupa $ABCCAC$. Njegovo ekvivalentno pakovanje predstavljeno
hromozomom grupa se enkodira kao:
\[
    A=\{0, 4\},\ B=\{1\},\ C=\{2, 3, 5\}.
\]
Ovo možemo zapisati i kao:
\[
    \{0, 4\},\ \{1\},\ \{2, 3, 5\}.
\]

Ovim pristupom smanjujemo stepen redudandnosti koji nastaje i kod naivnog 
pristupa i kod permutacija. Te će genetski algoritam implementirati grupe
kao jedinke.

\subsection{Operator ukrštanja}



\subsection{Operator mutacija}

\subsection{Računanje prilagodjenosti}

Kako problem predstavlja minimizaciju broja potrebnih kontejnera za pakovanje,
prirodno je za ocenu prilagodjenosti izabrati broj grupa u hromozomu.
Problem nastaje kada dve jedinke imaju istu ocenu prilagodjenosti, tj.\ 
broj kontejnera je isti za oba pakovanja. Kako odrediti koje pakovanje je
bolje od ta dva? Ovo prevazilazimo tako što merimo koliko je svaki od 
kontejnera dobro upakovan. Naime, dobro upakovan kontejner je onaj koji
ima malo neiskorišćenog kapaciteta.

Formalno, prilagodjenost računamo kao
\begin{equation}
    \text{fitness} = \frac{\sum_{j = 1}^N {(F_j/B)}^k}{N},
\end{equation}
gde je $N$ broj grupa u hromozomu, $B$ je kapacitet kontejnera, $F_j$
je napunjenost kontejnera $j$, i $k > 1$ je predefinisani koeficijent koji
pojačava odnose dobre upakovanosti. Eksperimentalo $k=2$ daje dobre 
rezultate\cite{f96}. 

Napunjenost kontejnera $j$ računamo kao 
\begin{equation}
    F_j = \sum_{i \in I_j} s(i),
\end{equation}
gde je je $I_j$ skup elemenata iz grupe $j$, a $s(i)$ veličina elementa $i$.

\subsection{Ostali parametri genetskog algoritma}

\section{Rezultati}

Algoritam je pokrenut na računaru sa sledećim specifikacijama: 
CPU --- Intel Core i5--2520M @ $4 \times 3.2GHz$,
RAM --- 7836MiB,
Kernel --- x86\_64 Linux 5.16.15-arch1--1,
g++ --- (GCC) 11.2.0.


\section{Zaključak}

Ovo je zakljucak.

\nocite{*}

\printbibliography[title={Literatura}]

\end{document}
